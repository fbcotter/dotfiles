% Allow latex commands
\usepackage{ltxcmds}

%%%%%%%%%%%%%%%%%%%%%%%%%%%%%%%%%%%%%%%%%%
% Layout - comment out if using another style
% \usepackage[a4paper, hmargin=4cm, vmargin=4cm]{geometry}
% \usepackage{afterpage}
% \listfiles
% \usepackage{listings}
% Add colour to your text and images
%\usepackage{color} 
% Allow us to easily customize page setup
% \usepackage{fancyhdr}
%\sloppy
%%%%% Fancyhdr options %%%%%%%
% This clears old style settings
% \fancyhead{}
% \fancyfoot{}
% \rfoot{\thepage}
% \pagestyle{fancy}
% \renewcommand{\headrulewidth}{0pt}
% \renewcommand{\footrulewidth}{0pt}
%The structure is: \iffloatpage{value for float page}{value for other pages}
%%%%%%%%%%%%%%%%%%%%%%%%%%%%%%%%%%%%%%%%%%

%%%%%%%%%%%%%%%%%%%%%%%%%%%%%%%%%%%%%%%%%%
% Other misc packages
\usepackage{ifpdf}
\usepackage{pdflscape}
\usepackage[table]{xcolor}
\usepackage{xspace} % for putting spaces after macros
% Use natbib for bibliography type
\usepackage[numbers]{natbib}
%\usepackage{babel}  % For multilingual support
%%%%%%%%%%%%%%%%%%%%%%%%%%%%%%%%%%%%%%%%%%

%%%%%%%%%%%%%%%%%%%%%%%%%%%%%%%%%%%%%%%%%%
% Math options
\usepackage{bm,bbm}
\usepackage{mathtools} % Fixes some bugs in amsmath
\usepackage{amssymb}
% \usepackage{mathptmx} % Replaced the times package
% Keep the old calligraphic math font
\DeclareMathAlphabet{\mathcal}{OMS}{cmsy}{m}{n}
%%%%%%%%%%%%%%%%%%%%%%%%%%%%%%%%%%%%%%%%%%%

%%%%%%%%%%%%%%%%%%%%%%%%%%%%%%%%%%%%%%%%%%%
% Figure, Tables and Equations
% Use pdftex option for graphicx if we are a pdf
\usepackage{subfig}
\usepackage[labelfont=bf,textfont=it]{caption}
\usepackage{multicol}
\usepackage{booktabs} % for top and bottom rules in tables
\usepackage{tabularx} % for variable width columns in tables
\usepackage{multirow}
\ifpdf
   \usepackage[pdftex]{graphicx}
\else
   \usepackage{graphicx}
\fi
\usepackage[export]{adjustbox}[2011/08/13]
% Set equation numbers <chapter>.<section>.<index>
\numberwithin{equation}{section} 
% Prevent line breaking for inline equations 
% by setting the cost of it very high
\binoppenalty=\maxdimen
\relpenalty=\maxdimen
% define "struts", as suggested by Claudio Beccari in
% %    a piece in TeX and TUG News, Vol. 2, 1993.
\newcommand\Tstrut{\rule{0pt}{2.6ex}}         % = `top' strut
\newcommand\Bstrut{\rule[-0.9ex]{0pt}{0pt}}   % = `bottom' strut
\usepackage{array}
\newcolumntype{L}[1]{>{\raggedright\let\newline\\\arraybackslash\hspace{0pt}}m{#1}}
\newcolumntype{C}[1]{>{\centering\let\newline\\\arraybackslash\hspace{0pt}}m{#1}}
\newcolumntype{R}[1]{>{\raggedleft\let\newline\\\arraybackslash\hspace{0pt}}m{#1}}
%%%%%%%%%%%%%%%%%%%%%%%%%%%%%%%%%%%%%%%%%%%


%%%%%%%%%%%%%%%%%%%%%%%%%%%%%%%%%%%%%%%%%%%
% Include tikz and some subpackages
\usepackage{tikz,pgfplots}
\usetikzlibrary{matrix,positioning,arrows}
\usetikzlibrary{decorations.markings}
\usetikzlibrary{calc}
% Allow tikz images to be compiled once
% \usetikzlibrary{external}\tikzexternalize         
%%%%%%%%%%%%%%%%%%%%%%%%%%%%%%%%%%%%%%%%%%%

%%%%%%%%%%%%%%%%%%%%%%%%%%%%%%%%%%%%%%%%%%%
% Add ability for algorithms and examples in your document
% Package for displaying algorithms in a document as pseudo code
\usepackage{algorithm2e} % Replaced the algorithm package
\usepackage{amsthm}
\usepackage{mdframed}
\newmdtheoremenv[
  hidealllines=true,
  innerleftmargin=8pt,%
  innerrightmargin=8pt,%
  innertopmargin=12pt,%
  innerbottommargin=12pt,%
  backgroundcolor=blue!10,%
  skipbelow=\baselineskip,%
  skipabove=\baselineskip]{exmp}{Example}[section]
\newmdenv[linecolor=blue, backgroundcolor=green!10,skipbelow=\baselineskip,
          skipabove=\baselineskip]{goals}
%%%%%%%%%%%%%%%%%%%%%%%%%%%%%%%%%%%%%%%%%%%

%%%%%%%%%%%%%%%%%%%%%%%%%%%%%%%%%%%%%%%%%%%
% Some Hyperref Options 
\usepackage{nameref}
\usepackage{hyperref}
\hypersetup{%
  colorlinks   = true,
  citecolor    = blue  
}
% Allow hyperlinks in our document. Redefine 
% the section names to use the squigly S
\renewcommand{\sectionautorefname}{\S}
\renewcommand{\subsectionautorefname}{\S}
\renewcommand{\subsubsectionautorefname}{\S}
\renewcommand{\pageautorefname}{p.}
\renewcommand{\chapterautorefname}{Chapter}
\newcommand{\subfigureautorefname}{\figureautorefname}
%%%%%%%%%%%%%%%%%%%%%%%%%%%%%%%%%%%%%%%%%%%

